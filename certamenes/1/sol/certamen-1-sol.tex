\documentclass[english, spanish, fleqn]{article}
\usepackage{babel}
\usepackage[utf8]{inputenc}
\usepackage{amsmath, amssymb, fourier}
\usepackage[ruled, noline]{algorithm2e}
\usepackage{pgf}

%%%
%%% For algorithm2e
%%%

\SetAlgorithmName{Algoritmo}{Algoritmo}{Índice de algoritmos}

\SetAlCapSty{mdseries}
\SetKwProg{Procedure}{procedure}{}{}
\SetKwProg{Function}{function}{}{}

\begin{document}
\thispagestyle{empty}
  \begin{center}
    {\huge\textbf{Pauta de Corrección\\[0.4\baselineskip]
		  Primer Certamen\\[0.3\baselineskip]
		  Algoritmos y Complejidad}}\\
    \vspace{2ex}
    {\large 27 de octubre de 2018}
  \end{center}
  \vspace{5mm}

  \begin{enumerate}
  \item % 20182c1p1
    Ambas partes son independientes.
    \begin{enumerate}
    \item % 20182c1p1a
      Bajo el supuesto indicado,
      podemos plantear la ecuación para la aproximación \(x^+\):
      \begin{align*}
	\frac{x_{n + 2} - x^+}{x_{n + 1} - x^+}
	  &= \frac{x_{n + 1} - x^+}{x_n - x^+} \\
	(x_{n + 2} - x^+) (x_n - x^+)
	  &= (x_{n + 1} - x^+)^2 \\
	x_{n + 2} x_n - (x_{n + 2} + x_n) x^+ + (x^+)^2
	  &= x_{n + 1}^2 - 2 x_{n + 1} x^+ + (x^+)^2 \\
	x_{n + 2} x_n - x_{n + 1}^2
	  &= (x_{n + 2} - 2 x_{n + 1} + x_n) x^+
      \end{align*}
      Despejando:
      \begin{equation*}
	x^+
	  = \frac{x_{n + 2} x_n - x_{n + 1}^2}{x_{n + 2} - 2 x_{n + 1} + x_n}
      \end{equation*}
      Esta fórmula es muy inestable numéricamente
      (el resultado es dividir dos expresiones
       que restan cantidades muy parecidas),
      puede escribirse en forma más estable como:
      \begin{equation*}
	x^+
	  = x_n - \frac{(\Delta x_n)^2}{\Delta^2 x_n}
      \end{equation*}
      donde:
      \begin{align*}
	\Delta x_n
	  &= x_{n + 1} - x_n \\
	\Delta^2 x_n
	  &= \Delta (\Delta x_n) \\
	  &= \Delta (x_{n + 1} - x_n) \\
	  &= x_{n + 2} - 2 x_{n + 1} + x_n
      \end{align*}
      Por esta fórmula
      se le conoce como el \emph{proceso \(\Delta^2\) de Aitken},
      nombrado
      por Alexander Aitken quien lo introdujo en 1926.
    \item % 20182c1p1b
      Sea cual sea el resultado de la primera parte de la pregunta,
      resulta el algoritmo~\ref{alg:20182c1p1b}
      para aproximar el punto fijo de \(g(x)\),
      con el punto de partida \(x_0\),
      (reemplace la fórmula para \(x^+\)
       a asignar a \(x_0\)).
      \begin{algorithm}
	\DontPrintSemicolon

	\Function{\(\operatorname{Aitken}(g, x_0, \epsilon)\)}{
	  \Repeat{\(\lvert x_2 - x_0 \rvert \le \epsilon\)}{
	    \(x_1 \leftarrow g(x_0)\) \;
	    \(x_2 \leftarrow g(x_1)\) \;
	    \(x_0 \leftarrow x_0 - (\Delta x_0)^2 /\Delta^2 x_0\) \;
	  }
	  \Return \(x_0\) \;
	}
	\caption{Algoritmo usando aceleración de Aitken}
	\label{alg:20182c1p1b}
      \end{algorithm}
    \end{enumerate}

    \vspace*{2\baselineskip}
    \begin{minipage}{1.0\linewidth}
      {\Large\textbf{Puntajes}}\\[0.5\baselineskip]
      \begin{tabular}{l@{\;}lrr}
	\multicolumn{2}{l}{\textbf{Total}}		 &    & 30 \\
	a) & Derivar la aproximación \(x^+\)		 & 20 &	   \\
	b) & Algoritmo claro				 & 10 &
      \end{tabular}
    \end{minipage}
\pagebreak[4]
  \item % 20182p2
    Buscamos una fórmula gaussiana exacta para polinomios hasta grado~\(3\),
    por lo que requeriremos \(n = 2\) puntos.
    La teoría nos indica buscar una secuencia de polinomios ortogonales
    con función de peso \(w(x) = x\) sobre \([0, 1]\).
    La fórmula será del tipo:
    \begin{equation*}
      \int_0^1 x f(x) \, \mathrm{d} x
	\approx A_0 f(x_0) + A_1 f(x_1)
    \end{equation*}

    Usamos el proceso de Gram-Schmidt para construir una base ortogonal,
    partiendo con los vectores \(1, x, x^2\).
    El producto interno relevante es:
    \begin{equation*}
      \langle f, g \rangle
	= \int_0^1 w(x) f(x) g(x) \, \mathrm{d} x
    \end{equation*}
    \begin{align*}
      p_0(x)
	&= 1 \\
      p_1(x)
	&= x - \frac{\langle x, p_0(x) \rangle}
		    {\langle p_0(x), p_0(x) \rangle} p_0(x) \\
      p_2(x)
	&= x^2 - \frac{\langle x^2, p_1(x) \rangle}
		      {\langle p_1(x), p_1(x) \rangle} p_1(x)
	       - \frac{\langle x^2, p_0(x) \rangle}
		      {\langle p_0(x), p_0(x) \rangle} p_0(x)
    \end{align*}
    Nos interesan los ceros de \(p_2\),
    llamémosles \(x_0\) y \(x_1\).

    Si llamamos \(\ell_i(x)\) a los polinomios de Lagrange:
    \begin{equation*}
      \ell_i(x)
	= \prod_{\substack{0 \le k \le n \\
			   k \ne i}}
	    \frac{x - x_k}{x_i - x_k}
    \end{equation*}
    los coeficientes están dados por:
    \begin{align*}
      A_i
	= \int_0^1 x \ell_i(x) \, \mathrm{d} x
    \end{align*}

    Un amable sistema de álgebra simbólica
    (Maxima,
     ver el archivo \texttt{p2.mc} adjunto)
    entrega:
    \begin{align*}
      p_0(x)
	&= 1 \\
      p_1(x)
	&= x - \frac{2}{3} \\
      p_2(x)
	&= x^2 - \frac{6}{5} x + \frac{3}{10}
    \end{align*}
    Los ceros de \(p_2\) son:
    \begin{equation*}
      x_0
	= \frac{6 - \sqrt{6}}{10}
      \quad
      x_1
	= \frac{6 + \sqrt{6}}{10}
    \end{equation*}
    Con esto los coeficientes buscados son:
    \begin{equation*}
      A_0
	= \frac{9 + \sqrt{6}}{36}
      \quad
      A_1
	= \frac{9 - \sqrt{6}}{36}
    \end{equation*}

    \vspace*{2\baselineskip}
    \begin{minipage}{1.0\linewidth}
      {\Large\textbf{Puntajes}}\\[0.5\baselineskip]
      \begin{tabular}{lrr}
	\textbf{Total}					     &	  & 30 \\
	Planteo como integración gaussiana de \(2\) puntos   & 10 &    \\
	Explicar cómo obtener \(p_2(x)\)		     & 10 &    \\
	Nodos son los ceros de \(p_2\)			     &	4 &    \\
	Fórmulas para coeficientes			     &	6 &
      \end{tabular}
    \end{minipage}
\pagebreak[4]
  \item % 20182c1p3
    Por turno.
    \begin{enumerate}
    \item % 20182c1p3a
      La idea deberá ser:
      \begin{itemize}
      \item
	Mover los \(n - 1\) platos superiores de \(A\) a \(C\)
      \item
	Mover el plato mayor de \(A\) a \(B\)
      \item
	Mover los \(n - 1\) platos superiores de \(C\) a \(A\)
      \item
	Mover el plato mayor de \(B\) a \(C\)
      \item
	Mover los \(n - 1\) platos superiores de \(A\) a \(C\)
      \end{itemize}
      Escrito formalmente como algoritmo recursivo
      resulta similar al del apunte,
      vea el algoritmo~\ref{alg:20182c1p3}.
      Se invoca \(\operatorname{Hanoi}(n, A, C)\).
      \begin{algorithm}[ht]
	\DontPrintSemicolon

	\Procedure{\(\operatorname{Hanoi}(n,
			 \mathit{src}, \mathit{dst})\)}{
	  \If{\(n > 0\)}{
	    \(\operatorname{Hanoi}(n - 1,
		  \mathit{src}, \mathit{dst})\) \;
	    Mover de \(\mathit{src}\) a \(B\) \;
	    \(\operatorname{Hanoi}(n - 1,
		  \mathit{dst}, \mathit{src})\) \;
	    Mover de \(B\) a \(\mathit{dst}\) \;
	    \(\operatorname{Hanoi}(n - 1,
		  \mathit{src}, \mathit{dst})\) \;
	  }
	}
	\caption{Solución recursiva a la variante de las torres de Hanoi}
	\label{alg:20182c1p3}
      \end{algorithm}
    \item % 20182c1p3b
      Llamemos \(H_n\) al número de movidas en esta variante
      de las torres de Hanoi.
      Tenemos la recurrencia:
      \begin{equation*}
	H_{n + 1}
	  = 2 + 3 H_n
	\quad
	H_0
	  = 0
      \end{equation*}

      Definiendo:
      \begin{equation*}
	h(z)
	  = \sum_{n \ge 0} H_n z^n
      \end{equation*}
      obtenemos la ecuación:
      \begin{equation*}
	\frac{h(z) - H_0}{z}
	  = 3 h(z) + \frac{2}{1 - z}
      \end{equation*}
      Nuestro sistema de álgebra simbólica nos da
      \(h\) como fracciones parciales
      (ver el archivo \texttt{p3.mc} adjunto):
      \begin{equation*}
	h(z)
	  = \frac{1}{1 - 3 z} - \frac{1}{1 - z}
      \end{equation*}
      de donde leemos:
      \begin{equation*}
	H_n
	  = 3^n - 1
      \end{equation*}
    \end{enumerate}

    \vspace*{2\baselineskip}
    \begin{minipage}{1.0\linewidth}
      {\Large\textbf{Puntajes}}\\[0.5\baselineskip]
      \begin{tabular}{l@{\;}lrr}
	\multicolumn{2}{l}{\textbf{Total}}		 &    & 35\\
	a) & Estrategia para resolver el problema	 & 20 &	  \\
	b) & Derivar la recurrencia			 & 15 &
      \end{tabular}
    \end{minipage}
\pagebreak[4]
  \item % 20182c1p4
    Este es un caso clásico
    de aplicación de \emph{\foreignlanguage{english}{backtracking}}.
    Usaremos la notación estándar de grafos,
    en particular \(N(v)\) son los vértices vecinos a \(v\).
    Resulta el algoritmo~\ref{alg:20182c1p4}.
      \begin{algorithm}[ht]
	\DontPrintSemicolon

	\Function{\(\operatorname{Paths}(G, s, t)\)}{
	  \If{\(s = t\)}{
	    \Return 0 \;
	  }
	  \(c \leftarrow 0\) \;
	  \For{\(v \in N(s)\)}{
	    \(c \leftarrow c + \operatorname{Paths}(G, v, t)\) \;
	  }
	  \Return \(c\) \;
	}
	\caption{Contar número de caminos en un DAG}
	\label{alg:20182c1p4}
      \end{algorithm}


    \vspace*{2\baselineskip}
    \begin{minipage}{1.0\linewidth}
      {\Large\textbf{Puntajes}}\\[0.5\baselineskip]
      \begin{tabular}{l@{\;}lrr}
	\multicolumn{2}{l}{\textbf{Total}}		     &	  & 35 \\
	Propuesta de backtracking			     & 10 &    \\
	Algoritmo detallado				     & 25 &
      \end{tabular}
    \end{minipage}
  \end{enumerate}
  \vfill\hfill HvB/\LaTeXe
\end{document}

%%% Local Variables:
%%% mode: latex
%%% TeX-master: t
%%% End:
