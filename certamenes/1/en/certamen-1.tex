\documentclass[english, spanish, fleqn]{article}
\usepackage{pgf}
\usepackage[ruled, noline, noend]{algorithm2e}
\usepackage{amsmath, amsthm, amssymb, fourier, mathrsfs}
\usepackage{enumitem}
\usepackage{babel}
\usepackage[utf8]{inputenc}
\usepackage[top=2cm, bottom=1.5cm, left=2cm, right=2cm]{geometry}

%%
%%% AMS-LaTeX theorem stuff
%%%

\theoremstyle{plain}
\newtheorem{proposition}{Proposición}

\begin{document}
\pagestyle{empty}
  \begin{center}
    {\huge\textbf{Primer Certamen\\[0.25\baselineskip]
		  Algoritmos y Complejidad}}\\
    \vspace{0.5\baselineskip}
    {\large 27 de octubre de 2018}
  \end{center}
  \vspace{2mm}
  \begin{center}
    \pgfimage[width=0.5\textwidth]{internal_monologues}
    % www.xkcd.org/2057
  \end{center}
  \vspace{2.5mm}
  \begin{enumerate}
  \item % 20182c1p1
    Considere la iteración de punto fijo
    \(x_{n + 1} = f(x_n)\),
    que converge linealmente a \(x^*\),
    vale decir hay \(0 < A < 1\) tal que:
    \begin{equation*}
      \lim_{n \to \infty} \frac{x_{n + 1} - x^*}{x_n - x^*}
	= A
    \end{equation*}
    \begin{enumerate}
    \item % 20182c1p1a
      \label{ques:20182c1p1a}
      Suponiendo que
      \((x_{n + 2} - x^*) / (x_{n + 1} - x^*)
	  \approx (x_{n + 1} - x^*) / (x_n - x^*)\),
      derive una aproximación \(x^+\) a \(x^*\).
    \item % 20182c1p1b
      Explique mediante un algoritmo
      cómo usar la aproximación de~\ref{ques:20182c1p1a}
      para acelerar la convergencia.
    \end{enumerate}
    \hspace*{\fill}(30 puntos)
  \item % 20182c1p2
    El Inspector Legrasse
    necesita calcular numéricamente integrales del tipo siguiente
    en forma exacta para \(f\) cúbicas:
    \begin{equation*}
      \int_0^1 x f(x) \, \mathrm{d} x
    \end{equation*}
    Diga cómo cumplir con el mínimo trabajo
    (evaluaciones de la complicada función \(f\)),
    dando las técnicas empleadas y las fórmulas precisas
    (no es necesario calcular los valores).
    \\ \hspace*{\fill}(30 puntos)
  \item % 20182c1p4
    Considere el problema de las torres de Hanoi,
    donde queremos mover \(n\) discos de la aguja \(A\) a la \(C\),
    pero con la restricción que toda movida debe involucrar la aguja \(B\)
    (por ejemplo,
     no podemos simplemente mover el disco mayor de \(A\) a \(C\),
     debemos pasar por \(B\)).
    \begin{enumerate}
    \item % 20182c1p4a
      Plantee una técnica
      (algoritmo)
      para resolver esta variante.
    \item % 2082c1p4b
      Halle una recurrencia para el número de platos movidos
      por su algoritmo en términos de \(n\).
    \end{enumerate}
    \hspace*{\fill}(35 puntos)
  \item % 20182c1p5
    Se da un DAG
    (grafo dirigido acíclico)
    \(G = (V, E)\),
    un vértice origen \(s\)
    y un vértice destino \(t\).
    Dé un algoritmo
    que entregue el número de caminos de \(s\) a \(t\) en \(G\).
    \\ \hspace*{\fill}(35 puntos)
  \end{enumerate}
  \vspace*{\fill}\hspace*{\fill}HvB/\LaTeXe
\end{document}

%%% Local Variables:
%%% mode: latex
%%% TeX-master: t
%%% End:
