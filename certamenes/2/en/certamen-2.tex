\documentclass[english, spanish, fleqn]{article}
\usepackage{pgf}
\usepackage[ruled, noline]{algorithm2e}
\usepackage{amsmath, amsthm, amssymb, fourier, mathrsfs}
\usepackage{enumitem}
\usepackage{babel}
\usepackage[utf8]{inputenc}
\usepackage[top=2.5cm, bottom=2cm, left=2cm, right=2cm]{geometry}

\newtheorem*{theorem}{Teorema}

\SetKwProg{Function}{function}{}{}
\SetKwProg{Procedure}{procedure}{}{}

\DeclareMathOperator{\var}{var}
\DeclareMathOperator{\Exp}{\mathbb{E}}

\begin{document}
\pagestyle{empty}
  \begin{center}
    {\huge\textbf{Segundo Certamen\\[0.25\baselineskip]
                  Algoritmos y Complejidad}}\\
    \vspace{0.5\baselineskip}
    {\large 22 de diciembre de 2018}
  \end{center}
  \vspace{2mm}
  \begin{center}
    \pgfimage[width=0.65\textwidth]{airplanes_and_spaceships}
    % www.xkcd.org/2074
  \end{center}
  \vspace{2.5mm}
  \begin{enumerate}
  \item % 20172c2p1
    Un vehículo parte del punto \(0\) y debe llegar al punto \(F\).
    Tiene un estanque con capacidad para viajar \(d\)~kilómetros sin recargar.
    Dadas posiciones \(s_1, s_2, \dotsc, s_n\) de estaciones de servicio,
    demuestre que el algoritmo
    de recargar combustible cada vez en la estación más lejana
    que puede alcanzar
    minimiza el número de paradas.
    \\ \hspace*{\fill}(25 puntos)
  \item % 20182c2p
    Para cierto problema de tamaño \(n\)
    cuenta con tres algoritmos alternativos.
    ¿Cuál elije si \(n\) es grande,
    y porqué?
    \begin{description}
    \item[Algoritmo A:]
      Ordena los datos dados
      (el costo de esto es \(\Theta(n \log n)\)),
      divide en \(3\) problemas de tamaño \(n / 3\)
      y combina en tiempo constante.
    \item[Algoritmo B:]
      Resuelve recursivamente \(2\) problemas de tamaño \(n - 1\) y \(n - 2\)
      y combina las soluciones en tiempo constante.
    \item[Algoritmo C:]
      Divide el problema de tamaño \(n\)
      en \(5\) problemas de tamaño \(n / 4\)
      y combina las soluciones en tiempo \(\Theta(n^2)\).
    \end{description}
    \hspace*{\fill}(35 puntos)
  \item % 20182c2p3
    En un río hay ciudades
    en las posiciones \(a_1, \dotsc, a_n\) en su ribera izquierda
    y \(b_1, \dotsc, b_n\) en la ribera derecha.
    Hay que conectar el máximo de ciudades con puentes,
    si se permite solo conectar \(a_i\) con \(b_i\)
    de manera que no se crucen puentes
    (las posiciones en ambas riberas
     no necesariamente vienen en orden).
    Plantee programación dinámica
    (la recurrencia con sus condiciones de borde,
     cómo almacenar los resultados intermedios,
     cuál es el resultado final
     y el orden de cálculo)
    para resolver este problema.
    \\ \hspace*{\fill}(30 puntos)
  \item % 20182c2p4
    Alice, Bob y Charlie discuten
    sobre cómo elegir al azar con igual probabilidad entre \(1, 2, 3\).
    Proponen lanzar una moneda
    que da cara
    (\emph{\foreignlanguage{english}{heads}}, \(H\))
    con probabilidad \(p\)
    o sello (\emph{\foreignlanguage{english}{tails}}, \(T\)):
    \begin{description}
    \item[Alice:]
      Lanzar dos veces,
      responder \(1\) si resulta \(H H\),
      \(2\) si es \(T H\),
      \(3\) para \(H T\),
      y repetir si resulta \(T T\).
    \item[Bob:]
      Lanzar dos veces,
      responder el número de \(H\) más uno.
    \item[Charlie:]
      Lanzar tres veces.
      Si sale una única \(H\),
      responder en qué lanzamiento apareció.
      Si no,
      repetir.
    \end{description}
    ¿Para qué valores de \(0 < p < 1\),
    si los hay,
    es correcto cada uno de los tres?
    \\ \hspace*{\fill}(30 puntos)
  \item % 20182c2p5
    La estructura \emph{fila ordenada}
    maneja operaciones \(\operatorname{orderedpush}(x)\),
    que elimina los elementos menores a \(x\) del inicio
    y los reemplaza por \(x\),
    y la operación \(\operatorname{pop}()\),
    que elimina el primer elemento de la fila.
    Si se construye mediante una lista enlazada,
    agregando y eliminando del principio,
    demuestre que las operaciones tienen costo amortizado \(O(1)\).
    \\ \hspace*{\fill}(30 puntos)
  \end{enumerate}
  \vspace*{\fill}\hspace*{\fill}HvB/\LaTeXe
\pagebreak[4]
\begin{center}
  \huge\textbf{Torpedo}
\end{center}
\section*{Teorema maestro}
\label{sec:teorema-maestro}

  Al dividir un problema de tamaño \(n\)
  en \(a\) problemas de tamaño \(n / b\)
  que se resuelven recursivamente,
  si el trabajo de dividir y combinar soluciones es \(f(n) > 0\),
  el tiempo para resolver un problema de tamaño \(n\) es,
  para \(\alpha = \log_b a\):
  \begin{equation*}
    t(n)
      = a t(n / b) + f(n)
    \quad t(1) = t_1
  \end{equation*}
  cuya solución es:
  \begin{equation*}
    t(n)
      = \begin{cases}
          \Theta(n^\alpha)
             & \text{\(f(n) = O(n^c)\) para \(c < \alpha\)} \\
          \Theta(n^\alpha)
             & \text{\(f(n) = \Theta(n^\alpha \log^\beta n)\)
                     con \(\beta < -1\)} \\
          \Theta(n^\alpha \log \log n)
             & \text{\(f(n) = \Theta(n^\alpha \log^\beta n)\)
                     con \(\beta = -1\)} \\
          \Theta(n^\alpha \log^{\beta + 1} n)
             & \text{\(f(n) = \Theta(n^\alpha \log^\beta n)\)
                     con \(\beta > -1\)} \\
          \Theta(f(n))
            & \text{\(f(n) = \Omega(n^c)\) con \(c > \alpha\)
                    y \(a f(n / b) < k f(n)\) para \(n\) grande
                    con \(k < 1\)}
        \end{cases}
  \end{equation*}

\section*{Algunas series notables}
\label{sec:series}

  En lo que sigue,
  \(m, n \in \mathbb{N}\),
  \(k \in \mathbb{N}_0\)
  y \(\alpha, \beta \in \mathbb{C}\).
  \begin{equation*}
    \frac{1}{1 - a z}
      = \sum_{k \ge 0} a^k z^k
    \qquad
    \frac{1 - z^{n + 1}}{1 - z}
      = \sum_{0 \le k \le n} z^k
    \qquad
    \frac{z}{(1 - z)^2}
      = \sum_{k \ge 0} k z^k
  \end{equation*}
  \begin{equation*}
    (1 + z)^\alpha
      = \sum_{n \ge 0} \binom{\alpha}{n} z^n
	 \qquad \binom{\alpha}{n}
	     = \frac{\alpha^{\underline{n}}}{n!}
	     = \frac{\alpha (\alpha - 1) \dotsm (\alpha - n + 1)}{n!}
	 \qquad
    \binom{n}{k}
	     = \frac{n!}{k! (n - k)!}
  \end{equation*}
  \begin{equation*}
    \binom{1/2}{n}
      = \frac{(-1)^{n - 1}}{n 2^{2 n - 1}} \, \binom{2 n - 2}{n - 1}
	    \quad \text{\ (si \(n \ge 1\))}
      \qquad
    \binom{-1/2}{n}
      = \frac{(-1)^n}{2^{2 n}} \, \binom{2 n}{n} \qquad
    \binom{-n}{k}
      = (-1)^k \, \binom{n + k - 1}{n - 1}
  \end{equation*}
  \begin{equation*}
    \frac{1}{(1 - z)^{n + 1}}
      = \sum_{k \ge 0} \binom{n + k}{n} z^k
      = \frac{1}{n!}
	   \sum_{k \ge 0} (k + 1) (k + 2) \dotsm (k + n) z^k
      = \frac{1}{n!} \sum_{k \ge 0} (k + 1)^{\bar{n}} z^k
  \end{equation*}
  \begin{equation*}
    \sum_{n \ge 0} \binom{n}{k} z^n
      = \frac{z^k}{(1 - z)^{k + 1}}
  \end{equation*}
\section*{Números de Fibonacci:}

  \begin{align*}
    &F_{n + 2} = F_{n + 1} + F_n \quad F_0 = 0, F_1 = 1
      \qquad \sum_{n \ge 0} F_n z^n = \frac{z}{1 - z - z^2}
      \qquad \sum_{n \ge 0} F_{n + 1} z^n = \frac{1}{1 - z - z^2} \\
    &F_n = \frac{\tau^n - \phi^n}{\sqrt{5}}
      \qquad
      \tau = \frac{1 + \sqrt{5}}{2} \approx 1,61803
      \quad \phi = \frac{1 - \sqrt{5}}{2}
                 = 1 - \tau = -1 / \tau \approx -0,61803
  \end{align*}
\end{document}

%%% Local Variables:
%%% mode: latex
%%% TeX-master: t
%%% End:
