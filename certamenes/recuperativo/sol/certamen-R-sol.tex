\documentclass[english, spanish, fleqn]{article}
\usepackage{babel}
\usepackage[utf8]{inputenc}
\usepackage{amsmath, amssymb, fourier}
\usepackage[ruled, noline]{algorithm2e}
\usepackage{pgf}

%%%
%%% For algorithm2e
%%%

\SetAlgorithmName{Algoritmo}{Algoritmo}{Índice de algoritmos}

\SetAlCapSty{mdseries}
\SetKwProg{Procedure}{procedure}{}{}
\SetKwProg{Function}{function}{}{}

\begin{document}
\thispagestyle{empty}
  \begin{center}
    {\huge\textbf{Pauta de Corrección\\[0.4\baselineskip]
		  Certamen Recuperativo\\[0.3\baselineskip]
		  Algoritmos y Complejidad}}\\
    \vspace{2ex}
    {\large 28 de diciembre de 2018}
  \end{center}
  \vspace{5mm}

  \begin{enumerate}
  \item % 20182cRp1
    Las tres partes son independientes.
    \begin{enumerate}
    \item % 20182cRp1a
      Habiendo \(3\) coeficientes,
      esperamos una fórmula exacta hasta grado \(2\).
    \item % 20182cRp1b
      Para hallar los coeficientes,
      podemos plantear un sistema de ecuaciones.
      Sabemos que:
      \begin{align*}
        \int_-1^1 x^k \, \mathrm{d} x
          &= \left. \frac{x^{k + 1}}{k + 1} \right|_{-1}^1 \\
          &= \frac{2}{k + 1} [ 2 \mid k + 1 ]
      \end{align*}
      por lo que obtenemos:
      \begin{alignat*}{4}
        &A_{-}       &+& A_0 &+& A_{+}       &=& 2 \\
       -&A_{-} x_0   &&      &+& A_{+} x_0   &=& 0 \\
        &A_{-} x_0^2 &&      &+& A_{+} x_0^2 &=& \frac{2}{3} \\
       -&A_{-} x_0^3 &&      &+& A_{+} x_0^3 &=& 0
      \end{alignat*}
      De la ecuación para \(k = 1\) tenemos que \(A_{-} = A_{+}\),
      con lo que inesperadamente se cumple para \(k = 3\) también.
      En realidad,
      cumple para todo \(k\) impar.
      El método es exacto hasta grado \(3\).

      Con la ecuación para \(k = 0\) ahora tenemos que \(A_0 = 2 - 2 A_{+}\),
      la ecuación para \(k = 2\) indica:
      \begin{equation*}
        A_{+}
          = \frac{1}{3 x_0}
      \end{equation*}
      En resumen:
      \begin{align*}
        A_{-}
          &= \frac{1}{3 x_0} \\
        A_0
          &= 2 - \frac{2}{3 x_0} \\
        A_{+}
          &= \frac{1}{3 x_0}
      \end{align*}
      Es claro que no necesariamente cumple para \(k = 4\).
    \item % 20182cRp1c
      Si podemos elegir \(x_0\)
      (idealmente \(x_{-}\) y \(x_{+}\),
       los puntos no necesariamente son simétricos en general),
      tenemos \(5\) parámetros,
      podemos esperar una fórmula exacta hasta grado \(4\).
      Por la observación anterior,
      será exacta hasta grado \(5\).
      Tenemos dos caminos:
      \begin{itemize}
      \item
        Usar la teoría de cuadratura gaussiana,
        desarrollando los polinomios ortogonales del caso
        y obteniendo los nodos como sus ceros.
      \item 
        Extender el sistema de ecuaciones del punto anterior.
      \end{itemize}
      En nuestro caso es más fácil extender el sistema
      con la ecuación adicional:
      \begin{equation*}
        A_{-} x_0^4 + A_{+} x_0^4
          = \frac{2}{5}
      \end{equation*}
      Conocemos \(A_{-}, A_{+}\):
      \begin{equation*}
        \frac{1}{3 x_0} x_0^4 + \frac{1}{3 x_0} x_0^4
          = \frac{2}{5} \\
      \end{equation*}
      Directamente obtenemos:
      \begin{equation*}
        x_0
          = \sqrt[3]{\frac{3}{5}}
      \end{equation*}
      y resultan:
      \begin{align*}
        A_{-}
          &= A_{+}
           = \frac{\sqrt[3]{45}}{9} \\
        A_0
          &= \frac{18 - 2 \sqrt[3]{45}}{9}
      \end{align*}
    \end{enumerate}
    
  \vspace*{2\baselineskip}
  \begin{minipage}{1.0\linewidth}
    {\Large\textbf{Puntajes}}\\[0.5\baselineskip]
    \begin{tabular}{l@{\;}lrr}
      \multicolumn{2}{l}{\textbf{Total}}	       &    & 25 \\
      a) & Grado máximo                                &  5 &    \\
      b) & Obtención de coeficientes                   & 10 &    \\
      c) & Mejor valor de \(x_0\)                      & 10 &
    \end{tabular}
  \end{minipage}
\pagebreak[4]
  \item % 20182cRp2
    Vemos que interesa el largo del intervalo \([i, j - 1]\)
    en el tiempo de ejecución.
    Sea \(T(k)\) el tiempo de ejecución de \(\operatorname{SlowHeap}(i, j)\)
    cuando \(j - i = k\).
    Del algoritmo,
    la recurrencia exacta es:
    \begin{equation*}
      T(n)
	= T(\lfloor n / 2 \rfloor)
	   + T(\lceil n / 2 \rceil - 1) + \Theta(n)
    \end{equation*}
    Podemos aplicar el teorema maestro a la recurrencia aproximada:
    \begin{equation*}
      T(n)
	= 2 T(n/2) + \Theta(n)
    \end{equation*}
    Esto es \(a = 2, b = 2, f(n) = \Theta(n)\),
    el valor crítico es \(\alpha = \log_2 2 = 1\).
    Se aplica el cuarto caso del teorema maestro,
    con \(\beta = 0\):
    \begin{align*}
      T(n)
	&= \Theta(n^\alpha \log^{\beta + 1} n) \\
	&= \Theta(n \log n)
    \end{align*}

    \vspace*{2\baselineskip}
    \begin{minipage}{1.0\linewidth}
      {\Large\textbf{Puntajes}}\\[0.5\baselineskip]
      \begin{tabular}{lrr}
	\textbf{Total}					     &	  & 30 \\
	Plantear la recursión				     & 10 &    \\
	Aplicar el teorema maestro			     & 20 &
      \end{tabular}
    \end{minipage}
\pagebreak[4]
  \item % 20182cRp3
    Corresponde hacer un análisis amortizado.
    Como hay una única operación,
    lo más sencillo es usar el método agregado.
    Sea \(n\) el número máximo de widgets fabricados.
    Es claro que George Akeley hará un total de \(n\) viajes,
    el costo por este concepto es \(n v\).
    El número total de cambios de dígito es
    (el último dígito cambia \(n\) veces,
     el dígito de las decenas \(\lfloor n / 10 \rfloor\) veces,
     y así sucesivamente):
    \begin{align*}
      n + \lfloor n / 10 \rfloor + \lfloor n / 10^2 \rfloor + \dotsb
	&= \sum_{k \ge 0} \lfloor n / 10^k \rfloor \\
	&< \sum_{k \ge 0} n / 10^k \\
	&= n \sum_{k \ge 0} 10^{-k} \\
	&= n \cdot \frac{1}{1 - 10^{-1}} \\
	&= \frac{10 n}{9}
    \end{align*}
    El costo total de \(n\) operaciones está acotado por:
    \begin{equation*}
      n v + \frac{10 n c}{9}
	= n \left( v + \frac{10 c}{9} \right)
    \end{equation*}
    con lo que el costo amortizado por operación está acotado por
    \(v + 10 c / 9\).

    \vspace*{2\baselineskip}
    \begin{minipage}{1.0\linewidth}
      {\Large\textbf{Puntajes}}\\[0.5\baselineskip]
      \begin{tabular}{lrr}
	\textbf{Total}					     &	  & 30 \\
	Usar método agregado				     &	5 &    \\
	Costo de viajes					     &	5 &    \\
	Costo de cambios de dígito			     & 10 &    \\
	Cotas manejables				     &	5 &    \\
	Costo amortizado				     &	5 &
      \end{tabular}
    \end{minipage}
\pagebreak[4]
  \item % 20182cRp4
    Por turno.
    \begin{enumerate}
    \item % 20182cRp4a
      El tiempo de ejecución del algoritmo
      no varía mayormente con la instancia presentada,
      y la tasa de error es menor a \(1/2\).
      es un algoritmo de Monte~Carlo.
      
      Su error es unilateral,
      con preferencia verdadero
      (\emph{\foreignlanguage{english}{true biased}}).
    \item % 20182cRp4b
      En una corrida,
      la probabilidad de resultado correcto es \(1/n\),
      por lo que la probabilidad de falla es \(1 - 1/n\);
      en \(k\) corridas independientes la probabilidad de error en todas
      es:
      \begin{equation*}
        \left( 1 - \frac{1}{n} \right)^k
      \end{equation*}
      Esto se parece seductivamente a la cota y límite:
      \begin{equation*}
        \lim_{n \to \infty} \left( 1 - \frac{1}{n} \right)^n
          = \frac{1}{\mathrm{e}}
      \end{equation*}
      por lo que para \(n\) grande la probabilidad de respuesta correcta
      en \(n\) corridas independientes es:
      \begin{equation*}
        1 - \frac{1}{\mathrm{e}}
      \end{equation*}
    \end{enumerate}

  \vspace*{2\baselineskip}
  \begin{minipage}{1.0\linewidth}
    {\Large\textbf{Puntajes}}\\[0.5\baselineskip]
    \begin{tabular}{l@{\;}lrr}
      \multicolumn{2}{l}{\textbf{Total}}	       &    & 30 \\
      a) & Clasificar                                  & 10 &    \\
      b) & Análisis                                    & 20 &
    \end{tabular}
  \end{minipage}
  \end{enumerate}
  \vfill\hfill HvB/\LaTeXe
\end{document}

%%% Local Variables:
%%% mode: latex
%%% TeX-master: t
%%% End:
