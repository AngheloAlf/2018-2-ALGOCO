\documentclass[english, spanish, fleqn]{article}
\usepackage{pgf}
\usepackage[ruled, noline]{algorithm2e}
\usepackage{amsmath, amsthm, amssymb, fourier, mathrsfs}
\usepackage[basic]{complexity}
\usepackage{enumitem}
\usepackage{csquotes}
\usepackage{babel}
\usepackage[utf8]{inputenc}
\usepackage[top=2cm, bottom=1.5cm, left=2cm, right=2cm]{geometry}

\newtheorem*{theorem}{Teorema}

%%% Para algorithm2e
\SetKwProg{Function}{function}{}{}
\SetKwProg{Procedure}{procedure}{}{}

\DeclareMathOperator{\var}{var}
\DeclareMathOperator{\Exp}{\mathbb{E}}

\begin{document}
\pagestyle{empty}
  \begin{center}
    {\huge\textbf{Certamen Recuperativo\\[0.25\baselineskip]
                  Algoritmos y Complejidad}}\\
    \vspace{0.5\baselineskip}
    {\large 23 de agosto de 2018}
  \end{center}
  \vspace{2mm}
  \begin{center}
    \pgfimage[width=0.2\textwidth]{horror_movies}
    % www.xkcd.org/2056
  \end{center}
  \vspace{2.5mm}
  \begin{enumerate}
  \item % 20182cRp1
    Se desea una fórmula de cuadratura
    para un \(0 < x_0 \le 1\)
    de la forma:
    \begin{equation*}
      \int_{-1}^1 f(x) \, \mathrm{d} x
	\approx A_{-} f(-x_0) + A_0 f(0) + A_{+} f(x_0)
    \end{equation*}
    \begin{enumerate}
    \item % 20182cRp1a
      ¿Cuál es el grado máximo de un polinomio para el cual puede ser exacta
      si \(x_0\) es un valor dado?
    \item % 20182cRp1b
      Explique cómo obtener los valores de los coeficientes
      \(A_{-}, A_0, A_{+}\).
    \item % 20182cRp1c
      ¿Cómo obtiene el mejor valor posible de \(x_0\)?
      ¿Cuál es el grado máximo de un polinomio para el cual es exacta?
    \end{enumerate}
    \hspace*{\fill}(25 puntos)
  \item % 20182cRp2
    Un \emph{\foreignlanguage{english}{heap}} es un árbol binario
    en el cual cada elemento es menor que sus hijos.
    El algoritmo~\ref{alg:SlowHeap}
    construye uno del rango \(a[i..j]\) de un arreglo;
    se invoca \(\operatorname{SlowHeap}(0, n - 1)\).
    \begin{algorithm}
      \DontPrintSemicolon
      \Function{\(\operatorname{SlowHeap}(i, j)\)}{
	\If{\(i = j\)}{
	  \Return Nodo sin hijos conteniendo \(a[i]\) \;
	}
	Busque \(m\) en \(i \le m \le j\) tal que \(a[m]\)
	  es el mínimo del rango ; \quad
	Intercambie \(a[m]\) con \(a[j]\) \;
	\(p_l
	  \leftarrow \operatorname{SlowHeap}
		       (i, \lfloor (i + j - 1) / 2 \rfloor)\) ; \quad
	\(p_r
	  \leftarrow \operatorname{SlowHeap}
		       (\lfloor (i + j - 1) / 2 \rfloor + 1, j - 1)\) \;
	  \Return Nodo conteniendo \(a[j]\)
	     con hijos \(p_l\) y \(p_r\) \;
      }
      \caption{\(\operatorname{SlowHeap}\)}
      \label{alg:SlowHeap}
    \end{algorithm}
    Dé la recurrencia para el tiempo de ejecución
    y resuélvala.
    \\ \hspace*{\fill}(30 puntos)
  \item % 20182cRp3
    McWidget tiene carteles en varios sitios
    que muestran el número de \emph{\foreignlanguage{english}{widget}}
    producidos.
    George Akeley gasta \(v\) en el viaje para actualizar los carteles
    y \(d\) por cada dígito que cambia
    (o sea,
     cambiar de \(21\) a \(22\) cuesta \(v + d\),
     \(99\) a \(100\) cuesta \(v + 3 d\)).
    Halle el costo amortizado por ítem producido.
    \\ \hspace*{\fill}(30 puntos)
  \item % 20182cRp4
    Dan un algoritmo que responde \textquote{Sí} o \textquote{No}.
    Si responde \textquote{Sí},
    esto siempre es correcto;
    si responde \textquote{No}
    está correcto con probabilidad al menos \(1/n\)
    si la entrada es de tamaño \(n\).
    Este algoritmo toma tiempo \(\Theta(n^2)\) para entradas de tamaño \(n\).
    \begin{enumerate}
    \item % 20182c2p4a
      Clasifique este algoritmo y su error.
    \item % 20182c2p4b
      ¿Cuántas veces hay que ejecutarlo para garantizar probabilidad
      al menos \(1 - 1 / \mathrm{e}\) de que la respuesta es correcta?
    \end{enumerate}
    \hspace*{\fill}(30 puntos)
  \end{enumerate}
  \vspace*{\fill}\hspace*{\fill}HvB/\LaTeXe
\pagebreak[4]
\section*{Teorema maestro}
\label{sec:teorema-maestro}

  Al dividir un problema de tamaño \(n\)
  en \(a\) problemas de tamaño \(n / b\)
  que se resuelven recursivamente,
  si el trabajo de dividir y combinar soluciones es \(f(n) > 0\),
  el tiempo para resolver un problema de tamaño \(n\) es,
  para \(\alpha = \log_b a\):
  \begin{equation*}
    t(n)
      = a t(n / b) + f(n)
    \quad t(1) = t_1
  \end{equation*}
  cuya solución es:
  \begin{equation*}
    t(n)
      = \begin{cases}
          \Theta(n^\alpha)
             & \text{\(f(n) = O(n^c)\) para \(c < \alpha\)} \\
          \Theta(n^\alpha)
             & \text{\(f(n) = \Theta(n^\alpha \log^\beta n)\)
                     con \(\beta < -1\)} \\
          \Theta(n^\alpha \log \log n)
             & \text{\(f(n) = \Theta(n^\alpha \log^\beta n)\)
                     con \(\beta = -1\)} \\
          \Theta(n^\alpha \log^{\beta + 1} n)
             & \text{\(f(n) = \Theta(n^\alpha \log^\beta n)\)
                     con \(\beta > -1\)} \\
          \Theta(f(n))
            & \text{\(f(n) = \Omega(n^c)\) con \(c > \alpha\)
                    y \(a f(n / b) < k f(n)\) para \(n\) grande
                    con \(k < 1\)}
        \end{cases}
  \end{equation*}

\section*{Algunas series notables}
\label{sec:series}

  En lo que sigue,
  \(m, n \in \mathbb{N}\),
  \(k \in \mathbb{N}_0\)
  y \(\alpha, \beta \in \mathbb{C}\).
  \begin{equation*}
    \frac{1}{1 - a z}
      = \sum_{n \ge 0} a^n z^n
	 \qquad \frac{1 - z^{m + 1}}{1 - z}
	    = \sum_{0 \le n \le m} z^n
  \end{equation*}
  \begin{equation*}
    (1 + z)^\alpha
      = \sum_{n \ge 0} \binom{\alpha}{n} z^n
	 \qquad \binom{\alpha}{n}
	     = \frac{\alpha^{\underline{n}}}{n!}
	     = \frac{\alpha (\alpha - 1) \dotsm (\alpha - n + 1)}{n!}
	 \qquad
    \binom{n}{k}
	     = \frac{n!}{k! (n - k)!}
  \end{equation*}
  \begin{equation*}
    \binom{1/2}{n}
      = \frac{(-1)^{n - 1}}{n 2^{2 n - 1}} \, \binom{2 n - 2}{n - 1}
	    \quad \text{\ (si \(n \ge 1\))}
      \qquad
    \binom{-1/2}{n}
      = \frac{(-1)^n}{2^{2 n}} \, \binom{2 n}{n} \qquad
    \binom{-n}{k}
      = (-1)^k \, \binom{n + k - 1}{n - 1}
  \end{equation*}
  \begin{equation*}
    \frac{1}{(1 - z)^{n + 1}}
      = \sum_{k \ge 0} \binom{n + k}{n} z^k
      = \frac{1}{n!}
	   \sum_{k \ge 0} (k + 1) (k + 2) \dotsm (k + n) z^k
      = \frac{1}{n!} \sum_{k \ge 0} (k + 1)^{\bar{n}} z^k
  \end{equation*}
  \begin{equation*}
    \sum_{n \ge 0} \binom{n}{k} z^n
      = \frac{z^k}{(1 - z)^{k + 1}}
  \end{equation*}

\section*{Cotas en probabilidad}
\label{sec:cotas-probabilidad}

  \begin{theorem}[Markov]
    Si \(X\) es una variable aleatoria no negativa,
    y \(c > 0\):
    \begin{equation*}
      \Pr[X > c]
        \le \frac{\Exp[X]}{c}
    \end{equation*}
  \end{theorem}
  \begin{theorem}[Chebyshev]
    Si \(X\) es una variable aleatoria de media \(\Exp[X] = \mu\)
    y varianza \(\var[X] = \Exp[(X - \mu)^2] = \sigma^2\),
    para \(c > 0\):
    \begin{equation*}
      \Pr[(X - \mu)^2 > c \sigma]
        \le \frac{1}{c^2}
    \end{equation*}
  \end{theorem}
  \begin{theorem}[Chernoff]
    Sean \(X_i\) variables aleatorias,
    \(0 \le X_i \le 1\) para todo \(i\),
    definamos \(X = \sum X_i\)
    y \(\mu = \Exp[X]\).
    Si \(c > 1\) con \(\beta(u) = u \ln u - u + 1\) se cumplen:
    \begin{align*}
      \Pr[X \ge c \mu]
        &\le \mathrm{e}^{- \beta(c) \mu} \\
      \Pr[X \le \mu / c]
        &\le \mathrm{e}^{- \beta(1 / c) \mu} \\
    \end{align*}
  \end{theorem}

\section*{Un límite clásico}
\label{sec:un-limite-clasico}

\begin{equation*}
  \lim_{n \to \infty} \left( 1 + \frac{x}{n} \right)^n
    = \mathrm{e}^x
\end{equation*}
\end{document}

%%% Local Variables:
%%% mode: latex
%%% TeX-master: t
%%% End:
