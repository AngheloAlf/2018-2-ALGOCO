\documentclass[english, spanish, fleqn]{article}
\usepackage{fourier}
\usepackage{babel}
\usepackage{minted}
\usepackage[utf8]{inputenc}
\usepackage[colorlinks, urlcolor=blue]{hyperref}

\newcommand{\num}{3}

\title{Algoritmos y Complejidad\\
       Tarea \#\num \\
       ``Recursion is beautiful''}
\author{Algorithm Knaves}
\date{15 de octubre de 2018}

\begin{document}
\maketitle
\thispagestyle{empty}

  Un hermoso ejemplo de recursión
  es el programita de Pike para reconocer
  una versión recortada de expresiones regulares.
  Vea la
    \href{https://www.cs.princeton.edu/courses/archive/spr09/cos333/beautiful.html}{explicación}
  de Kernighan para detalles.
  Reconoce solo las siguientes construcciones:

  \begin{tabular}[h]{cl}
    c                      & Calza el caracter 'c'
                             (salvo los especiales a continuación) \\
    .                      & Calza cualquier caracter \\
    \textasciicircum       & Calza el comienzo
                             del \emph{\foreignlanguage{english}{string}} \\
    \$                     & Calza el final
                             del \emph{\foreignlanguage{english}{string}} \\
    \textasteriskcentered  & Cero o más ocurrencias del caracter anterior
  \end{tabular}

  El programa es el del listado~\ref{lst:20182t3}.
  \begin{listing}
    \inputminted[frame = lines]{C}{match.c}
    \caption{Código C para reconcer expresiones regulares de Pike}
    \label{lst:20182t3}
  \end{listing}

  \begin{enumerate}
  \item % 20182t3p1
    Escriba una versión rudimentaria de \mintinline{shell}{grep(1)},
    al que se le llama como:
    
    \qquad\texttt{20182t3} \emph{<expresión>} \emph{<archivo>}

    que escriba todas las líneas en que la \emph{<expresión>} calza.
    No escriba nada más.
    \\ \hspace*{\fill}(20 puntos)
  \item % 20182t3p2
    Modifique el código dado para agregar la operación '+'
    (una o más veces lo anterior).
    \\ \hspace*{\fill}(25 puntos)
  \item % 20182t3p3
    Modifique el código dado para agregar la operación '?'
    (cero o una vez lo anterior).
    \\ \hspace*{\fill}(25 puntos)
  \item % 20172p2t4
    Al programa con ambas operaciones adicionales
    agregue la posibilidad de citar un caracter especial,
    vale decir,
    escribir por ejemplo '\textbackslash?' para calzar un '?'.
    \\ \hspace*{\fill}(30 puntos)
  \end{enumerate}
  Entregue varias versiones del código de calce,
  una para cada pregunta.
  Explique los cambios hechos al original.

  Note que \mintinline{shell}{grep(1)}
  se llama \emph{exactamente} como se indica,
  y únicamente escribe las líneas que calza.
  Cualquier otra salida se considerará un error.
  
  Tenga cuidado,
  para el \emph{\foreignlanguage{english}{shell}} los caracteres '*' y '?'
  tienen significado especial,
  para evitar accidentes se recomienda poner sus expresiones entre apóstrofes
  al hacer pruebas.

\input{condiciones}
  \vfill\hfill LW/HvB/\LaTeXe
\end{document}

%%% Local Variables:
%%% mode: latex
%%% TeX-master: t
%%% End:
