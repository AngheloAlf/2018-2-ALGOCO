\documentclass[english, spanish, fleqn]{article}
\usepackage{fourier}
\usepackage{babel}
\usepackage{minted}
\usepackage[utf8]{inputenc}
\usepackage[colorlinks, urlcolor=blue]{hyperref}

\newcommand{\num}{3}

\title{Algoritmos y Complejidad\\
       Tarea \#\num \\
       ``Recursion is beautiful''}
\author{Algorithm Knaves}
\date{15 de octubre de 2018}

\begin{document}
\maketitle
\thispagestyle{empty}

  Un hermoso ejemplo de recursión
  es el programita de Pike para reconocer
  una versión recortada de expresiones regulares.
  Vea la
    \href{https://www.cs.princeton.edu/courses/archive/spr09/cos333/beautiful.html}{explicación}
  de Kernighan para detalles.
  Reconoce solo las siguientes construcciones:

  \begin{tabular}[h]{cl}
    c                      & Calza el caracter 'c'
                             (salvo los especiales a continuación) \\
    .                      & Calza cualquier caracter \\
    \textasciicircum       & Calza el comienzo
                             del \emph{\foreignlanguage{english}{string}} \\
    \$                     & Calza el final
                             del \emph{\foreignlanguage{english}{string}} \\
    \textasteriskcentered  & Cero o más ocurrencias del caracter anterior
  \end{tabular}

  El programa es el del listado~\ref{lst:20182t3}.
  \begin{listing}
    \inputminted[frame = lines]{C}{match.c}
    \caption{Código C para reconcer expresiones regulares de Pike}
    \label{lst:20182t3}
  \end{listing}

  \begin{enumerate}
  \item % 20182t3p1
    Escriba una versión rudimentaria de \mintinline{shell}{grep(1)},
    al que se le llama como:
    
    \qquad\texttt{20182t3} \emph{<expresión>} \emph{<archivo>}

    que escriba todas las líneas en que la \emph{<expresión>} calza.
    No escriba nada más.
    \\ \hspace*{\fill}(20 puntos)
  \item % 20182t3p2
    Modifique el código dado para agregar la operación '+'
    (una o más veces lo anterior).
    \\ \hspace*{\fill}(25 puntos)
  \item % 20182t3p3
    Modifique el código dado para agregar la operación '?'
    (cero o una vez lo anterior).
    \\ \hspace*{\fill}(25 puntos)
  \item % 20172p2t4
    Al programa con ambas operaciones adicionales
    agregue la posibilidad de citar un caracter especial,
    vale decir,
    escribir por ejemplo '\textbackslash?' para calzar un '?'.
    \\ \hspace*{\fill}(30 puntos)
  \end{enumerate}
  Entregue varias versiones del código de calce,
  una para cada pregunta.
  Explique los cambios hechos al original.

  Note que \mintinline{shell}{grep(1)}
  se llama \emph{exactamente} como se indica,
  y únicamente escribe las líneas que calza.
  Cualquier otra salida se considerará un error.
  
  Tenga cuidado,
  para el \emph{\foreignlanguage{english}{shell}} los caracteres '*' y '?'
  tienen significado especial,
  para evitar accidentes se recomienda poner sus expresiones entre apóstrofes
  al hacer pruebas.

% Condiciones generales de tareas de Estructuras Discretas, 2015
\section{Condiciones de entrega}

  \begin{itemize}
  \item
    La tarea se realizará \emph{individualmente}
    (esto es grupos de una persona),
    sin excepciones.
  \item
    La entrega debe realizarse vía \href{http://moodle.inf.utfsm.cl}{Moodle}
    en un \emph{tarball} en el área designada al efecto, bajo el formato
    \texttt{tarea-\num-\emph{rol}.tar.gz}
    (\texttt{rol} con dígito verificador y sin guión).
    Puede uzar otra compresión que maneja Moodle,
    como \texttt{xz(1)}.

    Dicho \emph{tarball} debe contener las fuentes en LaTeX
    (al menos \texttt{tarea-\num.tex})
    de la parte escrita de su entrega,
    además de un archivo \texttt{tarea-\num.pdf},
    correspondiente a la compilación de esas fuentes.
  \item
    En  caso de haber programas,
    su ejecutable \emph{debe} llamarse \texttt{tarea-\num},
    de haber varias preguntas solicitando programas,
    estos deben llamarse \texttt{tarea-\num-1},
    \texttt{tarea-\num-2},
    etc.
    Si hay programas compilados,
    incluya una \texttt{Makefile}
    que efectúe las compilaciones correspondientes.

    Los programas se evalúan según que tan claros
    (bien escritos)
    son,
    si se compilan y ejecutan sin errores o advertencias según corresponda.
    Parte del puntaje es por ejecución correcta con casos de prueba.
    Si el programa no se ciñe a los requerimientos de entrada y salida,
    la nota respectiva es cero.
  \item
    Además de esto,
    la parte escrita de la tarea debe en hojas de tamaño carta
    en Secretaría Docente de Informática (Piso 1, edificio F3).
  \item
    Tanto el \emph{tarball} como la entrega física
    deben realizarse el día indicado
    en \href{http://moodle.inf.utfsm.cl}{Moodle}.
    No entregar la parte escrita en papel
    o no entregar en formato electrónico correcto
    tiene un descuento de 50 puntos.

    Por cada día de atraso se descontarán 20 puntos.
    A partir del tercer día de atraso
    no se reciben más tareas,
    y la nota de la tarea es cero.
  \item
    Nos reservamos el derecho de llamar a interrogación
    sobre algunas de las tareas entregadas.
    En tal caso,
    la nota base
    (antes de descuentos por atraso y otros)
    es la de la interrogación.
    No presentarse a la interrogación sin justificación previa
    significa automáticamente nota cero.
  \end{itemize}

  \vfill\hfill LW/HvB/\LaTeXe
\end{document}

%%% Local Variables:
%%% mode: latex
%%% TeX-master: t
%%% End:
