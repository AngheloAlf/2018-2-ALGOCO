\documentclass[spanish, fleqn]{article}
\usepackage{babel}
\usepackage[utf8]{inputenc}
\usepackage{fourier}
\usepackage{icomma}
\usepackage{csquotes}
\usepackage{amsmath, amsfonts, amsthm, fourier}
\usepackage[colorlinks, urlcolor=blue]{hyperref}
\usepackage{graphicx}
\usepackage{listings}

\newcommand{\num}{1}
\title{Tarea \num\\
       \large Algoritmos y Complejidad\\[3ex]
       \emph{\textquote{La curiosidad mató al gato\ldots}}}
\author{Algorithm Knaves}
\date{2018-09-11}

\begin{document}

\maketitle

  Efectuamos el análisis de los métodos para hallar ceros
  para el caso de ceros simpĺes
  (vale decir,
   \(f'(x^*) \ne 0\)).
  Analice la convergencia de los métodos de (a)~bisección,
  (b)~\emph{\foreignlanguage{latin}{regula falsi}},
  (c)~Newton
  y (d)~secante
  para el caso de ceros dobles
  (es decir,
   \(f'(x^*) = 0\) y \(f''(x^*) \ne 0\)).

\input{condiciones}

\end{document}

%%% Local Variables:
%%% mode: latex
%%% TeX-master: t
%%% End:
