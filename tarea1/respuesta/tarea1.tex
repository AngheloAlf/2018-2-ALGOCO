\documentclass[spanish, fleqn]{article}
\usepackage[spanish]{babel}
\usepackage[utf8]{inputenc}
\usepackage{amsmath, amssymb}
%\usepackage{fancyhdr}
%\usepackage{lipsum}
%\usepackage{enumerate}
%\usepackage{multicol}
\usepackage[colorlinks, urlcolor=blue]{hyperref}
\usepackage{graphicx}
\graphicspath{ {images/} }
\usepackage{xcolor}
\usepackage{verbatim}
\usepackage{listings}
\usepackage{ mathrsfs }
\usepackage{wrapfig}
\usepackage{enumitem}
\usepackage{ dsfont }
\usepackage{afterpage}

\usepackage{tikz} %% Grafos
\usepackage{fancyhdr} %% Para las cosas en las esquinas de cada pagina

\usepackage[a4paper,bindingoffset=0.0in,left=0.60in,right=0.70in,top=0.7in,bottom=0.7in,footskip=.25in]{geometry}

\newcommand\tab[1][0.5cm]{\hspace*{#1}}
\newcommand{\overbar}[1]{\mkern 1.5mu\overline{\mkern-1.5mu#1\mkern-1.5mu}\mkern 1.5mu}

\newcommand{\cero}{0}
\newcommand{\sumMayorIgual}[2][\cero]{\sum_{n\geq#1}{#2}}

\newcommand{\eclosure}[1]{\epsilon\text{-closure}(\{#1\})}

\newcommand\blankpage{%
    \null
    \thispagestyle{empty}%
    \addtocounter{page}{-1}%
    \newpage}

\newcommand{\numeroTarea}{1}
\newcommand{\nombreTarea}{«\textit{La curiosidad mató al gato.}»}

\newcommand{\xstar}{x^\star}

\begin{document}

\lstset
{
    language=[LaTeX]TeX,
    breaklines=false,
    basicstyle=\tt\scriptsize,
    keywordstyle=\color{blue},
    identifierstyle=\color{magenta},
}

\title{INF-155: Informática Teórica \\
Tarea \#\numeroTarea\\
\nombreTarea
}
\author{\href{mailto:anghelo.carvajal.14@gmail.com}{Anghelo Carvajal} \\ 201473062-4
}

\maketitle

\pagenumbering{gobble} %% Desactiva la numeracion de paginas


\pagestyle{fancy}
\fancyhf{}
\lhead{Tarea \numeroTarea: \nombreTarea}

%% \chead{\thesection} % 

%% \rhead{\rightmark} % 
\rhead{\rightmark}

\lfoot{\LaTeXe}

%% \cfoot{} %

\rfoot{Página \thepage}

%  \thepage
% adds number of the current page.
%  \thechapter
% adds number of the current chapter.
%  \thesection
% adds number of the current section.
%  \chaptername
% adds the word "Chapter" in English or its equivalent in the current language.
%  \leftmark
% adds name and number of the current top-level structure (for example, Chapter for reports and books classes; Section for articles ) in uppercase letters.
%  \rightmark
% adds name and number of the current next to top-level structure (Section for reports and books; Subsection for articles) in uppercase letters.

\newpage

%% \section{Section}

\pagenumbering{arabic} %% Activa numeracion de paginas

\section{Pregunta 1}

Efectuamos el análisis de los métodos para hallar ceros para el caso de ceros simples (vale decir, $f'(x^*)\neq 0)$. Analice la convergencia de los métodos de

\begin{enumerate}
    \item[a)] Bisección.
    \item[b)] Regula falsi.
    \item[c)] Newton
    \item[d)] Secante para el caso de ceros dobles (es decir, $f'(x^*) = 0)$ y $f''(x^*)\neq 0)$.
\end{enumerate}

\subsection{Respuesta 1}

\begin{enumerate}
    \item[a)] Bisección.
    
    Supongamos que el intervalo inicial es $[a, b]$. 
    
    Luego de iterar $n$ veces, el intervalo sera $\frac{b-a}{2^n}$. 
    
    En base a lo que nos dice este metodo, nuestro error maximo ($x_n * x^\star$) sera $\frac{\frac{b-a}{2^n}}{2}$, lo mismo que $\frac{b-a}{2^{n+1}}$.
    
    Si calculamos la convergencia lineal, resulta:
    \begin{align*}
        \lim_{n\rightarrow\infty} \left| \frac{e_{n+1}}{e_n} \right| &= \frac{\frac{b-a}{2^{n+2}}}{\frac{b-a}{2^{n+1}}} \\ 
        &= \frac{b-a}{2*2^{n+1}} * \frac{2^{n+1}}{b-a} \\ 
        &= \frac{1}{2}
    \end{align*}
    
    Como este método tiene una razón $\frac{1}{2} < 1$, converge linealmente sin que crezca el error.
    
    
    \item[b)] Regula falsi.
    
    A partir de la formula tenemos:
    \begin{align*}
        x_{n+1} &= x_n - f(x_n)\frac{x_n - x_0}{f(x_n) - f(x_0)} \\
        x_{n+1} - \xstar &= x_n - \xstar - f(x_n)\frac{x_n - x_0}{f(x_n) - f(x_0)} \\
        e_{n+1} &= e_n - f(x_n)\frac{x_n - x_0}{f(x_n) - f(x_0)} \\
        \frac{e_{n+1}}{e_n} &= 1 - \frac{f(x_n)}{e_n}\frac{x_n - x_0}{f(x_n) - f(x_0)} \\
        \frac{e_{n+1}}{e_n} &= 1 - \frac{f(x_n)}{e_n}\frac{x_n - x_0 + \xstar - \xstar }{f(x_n) - f(x_0)} \\
        \frac{e_{n+1}}{e_n} &= 1 - \frac{f(x_n)}{e_n}\frac{e_n - e_0}{f(x_n) - f(x_0)}
    \end{align*}
    
    Si expandimos un poco la serie de Taylor para la función $f(x_n)$:
    \begin{align*}
        f(x_n) &= f(\xstar) + f'(\xstar)(x_n - \xstar) + O((x_n - \xstar)^2) \\ 
        f(x_n) &= 0 + f'(x_n - \xstar)(e_n) + O((x_n - \xstar)^2) \\ 
        f(x_n) &= f'(\xstar)(e_n) + O((x_n - \xstar)^2) = O((x_n - \xstar))
    \end{align*}
    
    Reemplazando en la ecuación anterior tenemos:
    \begin{align*}
        \frac{e_{n+1}}{e_n} &= 1 - \frac{e_n * f'(\xstar)  + O((x_n - \xstar)^2)}{e_n}\frac{e_n - e_0}{e_n*f'(\xstar) + O((x_n - \xstar)^2) - f(x_0)} \\
        \frac{e_{n+1}}{e_n} &= 1 - \frac{e_n * f'(\xstar)  + O((x_n - \xstar)^2)}{e_n}\frac{e_n - e_0}{ O((x_n - \xstar)) - f(x_0)} \\
        \frac{e_{n+1}}{e_n} &= 1 - \frac{e_n * f'(\xstar)  + O((x_n - \xstar)^2)}{e_n}\frac{e_0 - e_n}{f(x_0)}(1 + O(x_n - \xstar)) \\
        \frac{e_{n+1}}{e_n} &= 1 - f'(\xstar)  \frac{e_0}{f(x_0)}(1 + O(x_n - \xstar)) \\
        \frac{e_{n+1}}{e_n} &= 1 - f'(\xstar) \frac{e_0}{f(x_0)}
    \end{align*}
    
    Finalmente, para saber si este método convergerá, tan solo tenemos que calcular $1 - f'(\xstar) \frac{e_0}{f(x_0)}$ y que de un valor menor a $1$. Lo mismo que $f'(\xstar) \frac{e_0}{f(x_0)}$ sea mayor que $0$.
    
    \item[c)] Newton.
    Usando series de Taylor:
    \begin{align*}
        f(x^\star) &= f(x_n) + \frac{f'(x_n)(x^\star - x_n)}{1!} + \frac{f''(x_n)(x^\star - x_n)^2}{2!} + O((x^\star - x_n)^2) \\ 
        0 &= f(x_n) + f'(x_n)(x^\star - x_n) + \frac{f''(x_n)(x^\star - x_n)^2}{2} \text{ // Reemplazamos $f(x^\star) = 0$.} \\ 
        -f(x_n) &= f'(x_n)(x^\star - x_n) + \frac{f''(x_n)(x^\star - x_n)^2}{2} \text{ // Restamos $f(x_n)$ a toda la ecuación. } \\ 
        -\frac{f(x_n)}{f'(x_n)} &= x^\star - x_n + \frac{f''(x_n)(x^\star - x_n)^2}{2f'(x_n)} \text{ // Dividimos por $f'(x_n)$. } \\ 
        x_n -\frac{f(x_n)}{f'(x_n)} &= x^\star + \frac{f''(x_n)(x^\star - x_n)^2}{2f'(x_n)} \text{ // Sumamos $x_n$. } \\ 
        x_{n+1} - x^\star &= \frac{f''(x_n)(x^\star - x_n)^2}{2f'(x_n)} \text{ // Aplicamos la igualdad $x_{n+1} = x_n -\frac{f(x_n)}{f'(x_n)}$ y restamos $x^\star$. } \\ 
        \frac{x_{n+1} - x^\star}{(x^\star - x_n)^2} &= \frac{f''(x_n)}{2f'(x_n)} \text{ // Dividimos por $(x^\star - x_n)^2$. } \\ 
        \frac{e_{n+1}}{e_n^2} &= \frac{f''(x_n)}{2f'(x_n)} \text{ // Finalmente aplicamos $e_n = x_n - x^\star$. }
    \end{align*}
    
    Al analizar la convergencia cuadrática ($lim_{n\rightarrow\infty} \left| \frac{e_{n+1}}{e_n^2} \right| $), vemos que es lo mismo que calcular $lim_{n\rightarrow\infty} \left| \frac{f''(x_n)}{2f'(x_n)} \right| $ y que esto sea distinto a infinito.
    
    \item[d)] Secante para el caso de ceros dobles (es decir, $f'(x^*) = 0)$ y $f''(x^*)\neq 0)$.
    
    El método de la secante dice:
    \begin{equation*}
        x_{n+2} = x_{n+1} - f(x_{n+1})\frac{x_{n+1} - x_{n}}{f(x_{n+1}) - f(x_n)}
    \end{equation*}
    donde $x_0$ y $x_1$ son el intervalo de la función en cuestión.
    
    Sea $x^\star$, de modo que $f(x^\star) = 0$ y $e_n = (x^\star - x_n)$ el error en la iteración $n$, de modo que $x_n = (x^\star - e_n)$. Aplicamos lo anterior y tenemos:
    \begin{align*}
        x^\star - e_{n+2} &= x^\star - e_{n+1} - f(x_{n+1})\frac{x^\star - e_{n+1} - x^\star + e_{n}}{f(x_{n+1}) - f(x_n)} \\
        - e_{n+2} &= - e_{n+1} - f(x_{n+1})\frac{ - e_{n+1} + e_{n}}{f(x_{n+1}) - f(x_n)}  \\
        e_{n+2} &= e_{n+1} - f(x_{n+1})\frac{e_{n+1} - e_{n}}{f(x_{n+1}) - f(x_n)} 
    \end{align*}
    
    Si expandimos un poco la serie de Taylor para la función $f(x_n)$:
    \begin{align*}
        f(x_n) &= f(\xstar) + f'(\xstar)(x_n - \xstar)  f''(\xstar)\frac{(x_n - \xstar)^2}{2} + O((x_n - \xstar)^3) \\ 
        f(x_n) &= 0 + 0 +   f''(\xstar)\frac{(x_n - \xstar)^2}{2} + O((x_n - \xstar)^3) \\ 
        f(x_n) &= f''(\xstar)\frac{e_n^2}{2} + O(e_n^3) = O(e_n^2) \\ 
    \end{align*}
    
    Reemplazando en la ecuación anterior tenemos:
    \begin{align*}
        e_{n+2} &= e_{n+1} - (f''(\xstar)\frac{e_{n+1}^2}{2} + O(e_{n+1}^3))\frac{e_{n+1} - e_{n}}{(f''(\xstar)\frac{e_{n+1}^2}{2} + O(e_{n+1}^3)) - f(x_n)} \\
        e_{n+2} &= e_{n+1} - f''(\xstar)\frac{e_{n+1}^2}{2}\frac{e_{n+1} - e_{n}}{O(e_{n+1}^2) - f(x_n)} \\
        e_{n+2} &= e_{n+1} + f''(\xstar)\frac{e_{n+1}^2}{2}\frac{e_{n+1} - e_{n}}{f(x_n)} (1 + O(e_{n+1}^2) ) \\
        \frac{e_{n+2}}{e_{n+1}} &= 1 + f''(\xstar)\frac{e_{n+1}^2}{2e_{n+1}}\frac{e_{n+1} - e_{n}}{f(x_n)} (1 + O(e_{n+1}^2) ) \\
        \frac{e_{n+2}}{e_{n+1}} &= 1 + f''(\xstar)\frac{e_{n+1}}{2}\frac{e_{n+1} - e_{n}}{f''(\xstar)\frac{e_n^2}{2} + O(e_n^3)} (1 + O(e_{n+1}^2) ) \\
        \frac{e_{n+2}}{e_{n+1}} &= 1 + f''(\xstar)\frac{e_{n+1}}{2}\frac{e_{n+1} - e_{n}}{f''(\xstar)\frac{e_n^2}{2}} (1 + O(e_{n+1}^2) ) (1 + O(e_n^3)) \\
        \frac{e_{n+2}}{e_{n+1}} &= 1 + \frac{f''(\xstar)}{f''(\xstar)}\frac{2e_{n+1}}{2}\frac{e_{n+1} - e_{n}}{e_n^2} (1 + O(e_{n+1}^2) ) (1 + O(e_n^3)) \\
        \frac{e_{n+2}}{e_{n+1}} &= 1 + e_{n+1}\frac{e_{n+1} - e_{n}}{e_n^2} (1 + O(e_{n+1}^2) ) (1 + O(e_n^3)) \\
        \\
        \\
        \frac{e_{n+1}}{e_n} &= 1 - \frac{e_n * f'(\xstar)  + O((x_n - \xstar)^2)}{e_n}\frac{e_n - e_0}{e_n*f'(\xstar) + O((x_n - \xstar)^2) - f(x_0)} \\
        \frac{e_{n+1}}{e_n} &= 1 - \frac{e_n * f'(\xstar)  + O((x_n - \xstar)^2)}{e_n}\frac{e_n - e_0}{ O((x_n - \xstar)) - f(x_0)} \\
        \frac{e_{n+1}}{e_n} &= 1 - \frac{e_n * f'(\xstar)  + O((x_n - \xstar)^2)}{e_n}\frac{e_0 - e_n}{f(x_0)}(1 + O(x_n - \xstar)) \\
        \frac{e_{n+1}}{e_n} &= 1 - f'(\xstar)  \frac{e_0}{f(x_0)}(1 + O(x_n - \xstar)) \\
        \frac{e_{n+1}}{e_n} &= 1 - f'(\xstar) \frac{e_0}{f(x_0)}(1 + O(x_n - \xstar)) \\
        \frac{e_{n+1}}{e_n} &= 1 - f'(\xstar) \frac{e_0}{f(x_0)}
    \end{align*}
    
\end{enumerate}


\end{document}
