\documentclass[spanish, fleqn]{article}
\usepackage[spanish]{babel}
\usepackage[utf8]{inputenc}
\usepackage{amsmath, amssymb}
%\usepackage{fancyhdr}
%\usepackage{lipsum}
%\usepackage{enumerate}
%\usepackage{multicol}
\usepackage[colorlinks, urlcolor=blue]{hyperref}
\usepackage{graphicx}
\graphicspath{ {images/} }
\usepackage{xcolor}
\usepackage{verbatim}
\usepackage{listings}
\usepackage{ mathrsfs }
\usepackage{wrapfig}
\usepackage{enumitem}
\usepackage{ dsfont }
\usepackage{afterpage}

\usepackage{tikz} %% Grafos
\usepackage{fancyhdr} %% Para las cosas en las esquinas de cada pagina

\usepackage[a4paper,bindingoffset=0.0in,left=0.60in,right=0.70in,top=0.7in,bottom=0.7in,footskip=.25in]{geometry}

\newcommand\tab[1][0.5cm]{\hspace*{#1}}
\newcommand{\overbar}[1]{\mkern 1.5mu\overline{\mkern-1.5mu#1\mkern-1.5mu}\mkern 1.5mu}

\newcommand{\cero}{0}
\newcommand{\sumMayorIgual}[2][\cero]{\sum_{n\geq#1}{#2}}

\newcommand{\eclosure}[1]{\epsilon\text{-closure}(\{#1\})}

\newcommand\blankpage{%
    \null
    \thispagestyle{empty}%
    \addtocounter{page}{-1}%
    \newpage}
    
\definecolor{bggray}{rgb}{0.95,0.95,0.95}
\lstdefinestyle{mypy}{
  language=python,
  backgroundcolor=\color{bggray},
  basicstyle=\ttfamily\small\color{orange!70!black},
  frame=L,
  keywordstyle=\bfseries\color{green!40!black},
  commentstyle=\itshape\color{purple!40!black},
  identifierstyle=\color{blue},
  stringstyle=\color{red},
  numbers=left,
  showstringspaces=false,
  xrightmargin=5pt,
  xleftmargin=10pt
}
\lstdefinestyle{myC}{
  language=C,
  backgroundcolor=\color{bggray},
  basicstyle=\ttfamily\small\color{orange!70!black},
  frame=L,
  keywordstyle=\bfseries\color{green!40!black},
  commentstyle=\itshape\color{purple!40!black},
  identifierstyle=\color{blue},
  stringstyle=\color{red},
  numbers=left,
  showstringspaces=false,
  xrightmargin=5pt,
  xleftmargin=10pt
}
\renewcommand{\lstlistingname}{Código}   
    

\newcommand{\ramo}{INF-155: Algoritmos y complejidad}
\newcommand{\numeroTarea}{4}
\newcommand{\nombreTarea}{\textit{Slicing through the 90s}}

\newcommand{\xstar}{x^\star}

\begin{document}

\lstset
{
    language=[LaTeX]TeX,
    breaklines=false,
    basicstyle=\tt\scriptsize,
    keywordstyle=\color{blue},
    identifierstyle=\color{magenta},
}

\title{\ramo \\
Tarea \#\numeroTarea\\
\nombreTarea
}
\author{\href{mailto:anghelo.carvajal.14@gmail.com}{Anghelo Carvajal} \\ 201473062-4
}

\maketitle

\pagenumbering{gobble} %% Desactiva la numeracion de paginas


\pagestyle{fancy}
\fancyhf{}
\lhead{Tarea \numeroTarea: \nombreTarea}

\chead{\ramo} % 

%% \rhead{\rightmark} % 
%% \rhead{Pregunta \thesection}

\lfoot{\LaTeXe}

%% \cfoot{} %

\rfoot{Página \thepage}

%  \thepage
% adds number of the current page.
%  \thechapter
% adds number of the current chapter.
%  \thesection
% adds number of the current section.
%  \chaptername
% adds the word "Chapter" in English or its equivalent in the current language.
%  \leftmark
% adds name and number of the current top-level structure (for example, Chapter for reports and books classes; Section for articles ) in uppercase letters.
%  \rightmark
% adds name and number of the current next to top-level structure (Section for reports and books; Subsection for articles) in uppercase letters.

\newpage

%% \section{Section}

\pagenumbering{arabic} %% Activa numeracion de paginas

\section{Pregunta 1}

Describa informal (pero claramente) un algoritmo \textbf{voraz} para resolver este problema. Considere que siempre habrá mas de un edificio ($n > 1$) y que el costo máximo de un ``\textit{buen slice}'' es siempre mayor o igual al número de edificios ($k \geq n$). Debe dejar bien claro los pasos y decisiones que realiza su algoritmo, puede apoyarse en diagramas y demases que quepan en su informe.

\subsection{Respuesta 1}

\begin{enumerate}
    \item El algoritmo empezaría a \textit{nivel de piso}.
    
    \item Revisa si, en la \texttt{altura} en la que se encuentra, todos los edificios tienen la misma altura.
    
    \begin{enumerate}
        \item Si se cumple, subir a la siguiente \texttt{altura} y repetir el paso.
        
        \item Si no, ir al siguiente paso.
    \end{enumerate}
    
    \item Tomar los \texttt{pisos} de la \texttt{altura} actual (\texttt{$h_i$}) y considerarlos para el \textit{slice}, dejándolos en un \texttt{acumulado}. Luego entramos a un ciclo:
    
    \begin{enumerate}
        \item Entramos a la siguiente \texttt{altura}.
        
        \item Contamos la cantidad de \texttt{pisos} de dicha \texttt{altura}.
        
        \begin{enumerate}
            \item Si dicha cantidad mas la cantidad en \texttt{acumulado} no supera \texttt{$k$}, entonces agregamos estos \texttt{pisos} a \texttt{acumulado} y repetimos este ciclo (vamos a $3.a$).
            
            \item En caso contrario, nos devolvemos a la ultima \texttt{altura} que agregamos a nuestro \texttt{acumulado} y luego salimos del ciclo (vamos a $4$).
        \end{enumerate}
        
    \end{enumerate}
    
    \item Lo que tengamos en \texttt{acumulado} sera uno de nuestros \textit{slices}, guardándolo en \texttt{total}.
    
    \item Si queda mas \texttt{altura} por revisar, nos movemos a la siguiente altura y nos vamos al paso $3$.
    
    \item Si ya no queda mas altura, nuestros \texttt{slices} se encontraran en la pila LIFO \texttt{total}.
\end{enumerate}

En otras palabras, nuestro algoritmo va desde abajo hacia arriba, tomando la mayor cantidad de pisos por \texttt{slice} posible, siempre completando una cantidad entera de alturas.

\section{Pregunta 2}

Demuestre formalmente que su algoritmo encuentra el número mínimo de ``\textit{buenos slices}'', es decir un óptimo global. En caso de suceder, describa las restricciones y acotaciones para las cuales su algoritmo es óptimo, y describa las modificaciones para hacer su algoritmo óptimo para toda posible instancia del problema.

\subsection{Respuesta 2}

Para demostrar lo pedido, debemos demostrar lo siguiente:

\begin{enumerate}
    \item \textit{Greedy choice property}
    
    \tab Supongamos que existe una solución óptima $\Pi_{h}^{\star}$, la cual el primer \textit{slice} no es el de mayor cantidad de pisos $m$, entonces posee un subconjunto de \textit{slice}s que poseen una cantidad de pisos $m$, si reemplazamos esos \textit{slice}s tendríamos una solución aun mejor, pero $\Pi_{h}^{\star}$ ya era óptima, por lo cual llegamos a una contradicción.
    
    \item \textit{Inductive Substructure}
    
    \tab Al elegir los $m$ pisos para el \texttt{slice} $i$, resulta el subproblema $P_{h-m}$ para cortar los $h-m$ pisos restantes.
    El conjunto $\Pi_{h-m} \cup \{i\}$ es solución factible para $P_h$, ya que cortan a los edificios de la forma deseada y no hay restricciones externas al resolver $P_{h-m}$.
    
    \item \textit{Optimal Substructure}
    
    \tab Debemos demostrar que $\Pi_{h} = \Pi_{h-m} \cup \{i\}$ es una solucion optima. Es decir, no se puede resolver el problema con menos \texttt{slice}s que $|\Pi_{h-m}| + 1$.
    
    \tab Por demostración al absurdo, supongamos que podemos reemplazar el subconjunto $\Pi_{h-m}$ por otro que contenga una cantidad menor de \textit{slice}s.
    
    Haciendo este cambio, la cantidad de \textit{slice}s de $P_h$ seria $|\Pi_{h-m}|$. Pero por lo demostrado en \textit{Greedy choice}, existe una solución igual de óptima que tiene la tiene el \textit{slice} $i$, por lo tanto, si sacamos este \textit{slice} tendríamos una solución para $P_{h-m}$ que requeriría $|\Pi_{h-m}| - 1$ \textit{slices}, por lo cual $\Pi_{h-m}$ no seria la solución óptima, por lo cual llegamos a una contradicción.

\end{enumerate}

\end{document}
