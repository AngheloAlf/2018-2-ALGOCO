\section{Condiciones de Evaluación de código}
Se evaluará:
\begin{itemize}
    \item \textbf{Ejecución correcta}: que funcionen los casos de prueba y \textbf{NO} sea posible encontrar casos en que el programa entregue una respuesta equivocada.
    \item \textbf{Complejidad computacional adecuada}: Que el algoritmo implementado tenga una complejidad igual o mejor que la esperada, y que sea ad-hoc a la materia que se está evaluando (e.g. no utilizar programación dinámica si se pide programar un algoritmo voraz).
    \item \textbf{Calidad del programa}: uso adecuado de funciones, uso de estructuras de control, uso de estructuras de datos, código claro y simple.
    \item \textbf{Código ordenado}: nombres adecuados, indentación correcta, comentarios suficientes, ausencia de código comentado.
\end{itemize}
Para más información refiérase a las condiciones de evaluación del código publicadas en moodle
